%------------------------------------------
% Szakdolgozat
% 
% Android vezérlésű okos LED-rendszer fejlesztése
% Tar Dániel
%
% Konzulens: Szakály Norbert
%------------------------------------------

\documentclass[12pt,a4paper]{article}
\usepackage{misc_content/format}


\begin{document}

\pagenumbering{roman}
\author{\myname}
\title{Android vezérlésű okos LED-rendszer fejlesztése}

%--Szennycímoldal-----------------------------------------
\thispagestyle{empty}
\begin{center}
     \MakeUppercase{\myname}\\
     SZAKDOLGOZAT
\end{center}
\newpage

%--Sorozatcímoldal----------------------------------------
\thispagestyle{empty}
\begin{center}
     BUDAPESTI MŰSZAKI ÉS GAZDASÁGTUDOMÁNYI EGYETEM\\
     GÉPÉSZMÉRNÖKI KAR\\
     MECHATRONIKA, OPTIKA ÉS GÉPÉSZETI INFORMATIKA TANSZÉK\\[1ex]
     \resizebox{2.5cm}{!}{
          \includegraphics{resources/mogi.png}
     }\\[1ex]
     SZAKDOLGOZATOK
\end{center}
\newpage

%--Címoldal-----------------------------------------------
\thispagestyle{empty}
\begin{titlepage} %environment for unique titlepage design
\centering
\resizebox{4.5cm}{!}{
  \includegraphics{resources/bme_logo_kicsi.eps}
}\\[1ex]
{\bf BUDAPESTI MŰSZAKI ÉS GAZDASÁGTUDOMÁNYI EGYETEM}\\
{\bf GÉPÉSZMÉRNÖKI KAR}\\
{\bf MECHATRONIKA, OPTIKA ÉS GÉPÉSZETI INFORMATIKA TANSZÉK}\\[3cm]

{\LARGE \scshape \myname}\\[2ex]
{\LARGE SZAKDOLGOZAT}\\[2ex]
{\LARGE \bf Android vezérlésű okos LED-rendszer fejlesztése}\\[2ex]
{\itshape Development of android controlled wireless LED-lighting system}\\[5cm]

\begin{tabularx}{\textwidth}{XXXX}
Konzulens: & Témavezető: \\
\hspace{0.75cm} \itshape asd & \hspace{0.75cm} \itshape Szakály Norbert \\
\hspace{0.75cm} asd & \hspace{0.75cm} tanszéki mérnök \\
\end{tabularx}\\[5.5cm]

{\Large Budapest, 2018}
\end{titlepage}
\newpage

%--Záradék és nyilatkozatok-------------------------------
\subfile{misc_content/nyilatkozatok}
\newpage

%--Tartalomjegyzék----------------------------------------
\thispagestyle{plain}
\tableofcontents
\newpage

%--Előszó-------------------------------------------------
\thispagestyle{empty}
\subfile{sections/eloszo}
\newpage

%--Jelölések jegyzéke-------------------------------------
\thispagestyle{plain}
\section*{Jelölések jegyzéke}

A táblázatban a többször előforduló jelölések magyar és angol nyelvű elnevezése, valamint a fizikai mennyiségek es etén annak mértékegysége található. Az egyes mennyiségek jelölése – ahol lehetséges – megegyezik hazai és a nemzetközi szak-irodalomban elfogadott jelölésekkel. A ritkán alkalmazott jelölések magyarázata első előfordulási helyüknél található.
\newpage

%--Footer-------------------------------------------------
\thispagestyle{fancy}
\pagenumbering{arabic}

%--Bevezetés----------------------------------------------
\thispagestyle{fancy}
\subfile{sections/bevezetes}
\newpage

%--Irodalomkutatás----------------------------------------
%\subfile{sections/literature}
%\newpage
%---------------------------------------------------------

%--Tartalmi rész------------------------------------------
    \thispagestyle{fancy}
    
    % Okos LED-rendszerek felkutatása, forgalomban lévő eszközök áttekintése
    \subfile{sections/1_section_szakirodalom}
    
    % Tervezési feladat követelményeinek felállítása
    % \subfile{sections/2_section}
    
    % Eszköz (elektronikai és beágyazott szoftverének) tervezése és összeállítása
    % \subfile{sections/3_section}
    
    % Androidos alkalmazás elkészítése
    \subfile{sections/4_section_android_application}
    
    % Elkészült eszköz értékelése, költségterv számolása
    % \subfile{sections/5_section}
    \newpage
%---------------------------------------------------------

%--Összefoglalás------------------------------------------
\subfile{sections/osszefoglalas_HU}
\newpage
%---------------------------------------------------------

%--Források-----------------------------------------------
% \addcontentsline{toc}{section}{Források}
% \bibliographystyle{unsrt}
% \bibliographystyle{acm}
\bibliographystyle{plain}
\bibliography{references}
\newpage
%---------------------------------------------------------

%--Angol összefoglalás------------------------------------
\subfile{sections/osszefoglalas_EN}
\newpage
%---------------------------------------------------------

%--Rövidítések jegyzéke-----------------------------------
%\addcontentsline{toc}{section}{Rövidítések jegyzéke}
% \begin{abbreviations}
% 	\item[HRC]	\angol{Human Robot Collaboration}
% 	\item[IoT]	\angol{Internet of Things}
% 	\item[IoE]	\angol{Internet of Everything}
% 	\item[TCP]	\angol{Tool Center Point}
% 	\item[HRI]	\angol{Human-Robot Interaction}
% \end{abbreviations}
%---------------------------------------------------------

%--Függelékek---------------------------------------------
%\newpage
%\appendix
%\subfile{sections/appendix}
%---------------------------------------------------------
\end{document}