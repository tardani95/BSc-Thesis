%------------------------------------------
% Szakdolgozat
% 
% Android vezérlésű okos LED-rendszer fejlesztése
% Tar Dániel
%
% Konzulens: Szakály Norbert
%------------------------------------------

\documentclass[12pt,a4paper]{article}
%,twoside]{article}
\usepackage{misc_content/format}

\begin{document}

% \pagenumbering{roman}
\author{\myname}
\title{Android vezérlésű okos LED-rendszer fejlesztése}

%--Szennycímoldal-----------------------------------------
\thispagestyle{empty}
\begin{center}
     \MakeUppercase{\myname}\\
     SZAKDOLGOZAT
\end{center}
\newpage

%--Sorozatcímoldal----------------------------------------
\thispagestyle{empty}
\begin{center}
     BUDAPESTI MŰSZAKI ÉS GAZDASÁGTUDOMÁNYI EGYETEM\\
     GÉPÉSZMÉRNÖKI KAR\\
     MECHATRONIKA, OPTIKA ÉS GÉPÉSZETI INFORMATIKA TANSZÉK\\[1ex]
     \resizebox{2.5cm}{!}{
          \includegraphics{resources/mogi.png}
     }\\[1ex]
     SZAKDOLGOZATOK
\end{center}
\newpage

%--Címoldal-----------------------------------------------
\thispagestyle{empty}
\begin{titlepage} %environment for unique titlepage design
\centering
\resizebox{4.5cm}{!}{
  \includegraphics{resources/bme_logo_kicsi.eps}
}\\[1ex]
{\bf BUDAPESTI MŰSZAKI ÉS GAZDASÁGTUDOMÁNYI EGYETEM}\\
{\bf GÉPÉSZMÉRNÖKI KAR}\\
{\bf MECHATRONIKA, OPTIKA ÉS GÉPÉSZETI INFORMATIKA TANSZÉK}\\[3cm]

{\LARGE \scshape \myname}\\[2ex]
{\LARGE SZAKDOLGOZAT}\\[2ex]
{\LARGE \bf Android vezérlésű okos LED-rendszer fejlesztése}\\[2ex]
{\itshape Development of android controlled wireless LED-lighting system}\\[5cm]

\begin{tabularx}{\textwidth}{XXXX}
~~~~~~ & Témavezető: \\
\hspace{0.75cm} \itshape ~~~~~~ & \hspace{0.75cm} \itshape Szakály Norbert \\
\hspace{0.75cm} ~~~~~~ & \hspace{0.75cm} tanszéki mérnök \\
\end{tabularx}\\[5.5cm]

{\Large Budapest, 2018}
\end{titlepage}
\newpage

%--Záradék és nyilatkozatok-------------------------------
\pagenumbering{roman}
\setcounter{page}{4}
% \subfile{misc_content/nyilatkozatok}
\newpage
%--Tartalomjegyzék----------------------------------------

\newpage\pagestyle{plain}
\addtocontents{toc}{\protect\thispagestyle{plain}}
\tableofcontents
\newpage

% \newpage
% \thispagestyle{empty}
% \tableofcontents



% \pagenumbering{arabic}
%--Előszó-------------------------------------------------
\newpage
\thispagestyle{plain}
\subfile{sections/eloszo}
\newpage

%--Jelölések jegyzéke-------------------------------------
\thispagestyle{plain}
\section*{Jelölések jegyzéke}

A táblázatban a többször előforduló jelölések magyar és angol nyelvű elnevezése, valamint a fizikai mennyiségek esetén annak mértékegysége található. Az egyes mennyiségek jelölése – ahol lehetséges – megegyezik hazai és a nemzetközi szakirodalomban elfogadott jelölésekkel. A ritkán alkalmazott jelölések magyarázata első előfordulási helyüknél található.

% \begin{table}[h!]
% \centering
% \begin{tabular}{lll}
% Rövidítések &                                    &                                     \\
% \hline
% Jelölés     & Megnevezés                         & Értelmezés                          \\
% \hline
% CAD         & Computer-aided design              &                                     \\
% CAE         & Computer-aided Design              &                                     \\
% CAM         &                                    &                                     \\
% DC          & Direct current                     & Egyenáram                           \\
% IC          & Integrated Circuit                 & Intergált áramköt                   \\
% IDE         & Integrated development environment & Integrált fejlesztői környezet      \\
% LDO         & Low-dropout regulator              & Alacsony feszültségesésű szabályozó \\
% LED         & Light Emitting Diode               & fényt kibocsájtó dióda              \\
% LDO         & Low-dropout regulator              & Alacsony feszültségesésű szabályozó \\
% LDO         & Low-dropout regulator              & Alacsony feszültségesésű szabályozó \\
% LDO         & Low-dropout regulator              & Alacsony feszültségesésű szabályozó \\
% LDO         & Low-dropout regulator              & Alacsony feszültségesésű szabályozó \\
% LDO         & Low-dropout regulator              & Alacsony feszültségesésű szabályozó \\
% LDO         & Low-dropout regulator              & Alacsony feszültségesésű szabályozó \\
% LDO         & Low-dropout regulator              & Alacsony feszültségesésű szabályozó \\
% LDO         & Low-dropout regulator              & Alacsony feszültségesésű szabályozó \\
% LDO         & Low-dropout regulator              & Alacsony feszültségesésű szabályozó \\
% LDO         & Low-dropout regulator              & Alacsony feszültségesésű szabályozó
% \end{tabular}
% \end{table}

%Ennyit találtam most:
% LED
% UART
% IC
% NYÁK
% CAD
% CAM
% CAE
% DC/DC áramkör
% LDO
% LiPo akkumulátor
% ART
% DEX-fájl


\newpage

%--Footer-------------------------------------------------
\pagestyle{fancy}
\thispagestyle{fancy}
\pagenumbering{arabic}

%--Bevezetés----------------------------------------------
\thispagestyle{fancy}
\subfile{sections/bevezetes}
\newpage

%--Tartalmi rész------------------------------------------
    \thispagestyle{fancy}
    
    % Okos LED-rendszerek felkutatása, forgalomban lévő eszközök áttekintése
    \clearpage
    \subfile{sections/1_section_szakirodalom}
    
    % Tervezési feladat követelményeinek felállítása
    \clearpage
    \subfile{sections/2_section_tervezesi_kovetelmenyek}
    
    % Eszköz (elektronikai és beágyazott szoftverének) tervezése és összeállítása
    \clearpage
    \subfile{sections/3_section_eszkoz_tervezese_es_osszeallitasa}
    
    % Beagyazott szoftver keszitese
    \clearpage
    \subfile{sections/4_section_beagyazott_szoftver_keszitese}
    
    % Androidos alkalmazás elkészítése
    \clearpage
    \subfile{sections/5_section_android_application}
    
    % Elkészült eszköz értékelése, költségterv számolása
    \clearpage
    \subfile{sections/6_section}
    \newpage
%---------------------------------------------------------


%\pagenumbering{gobble} % oldalszamozas kikapcsolasa
%--Összefoglalás------------------------------------------
% \subfile{sections/osszefoglalas_HU}
% \newpage
%---------------------------------------------------------

%--Források-----------------------------------------------
% \addcontentsline{toc}{section}{Források}
% \bibliographystyle{unsrt}
% \bibliographystyle{acm}
% \bibliographystyle{plain}


\cleardoublepage
\phantomsection
\addcontentsline{toc}{section}{Hivatkozások}
\bibliographystyle{unsrt}
\bibliography{references}
%---------------------------------------------------------

%--Angol összefoglalás------------------------------------
 \cleardoublepage
 \subfile{sections/osszefoglalas_EN}
 \newpage
%---------------------------------------------------------

%--Rövidítések jegyzéke-----------------------------------
%\addcontentsline{toc}{section}{Rövidítések jegyzéke}
% \begin{abbreviations}
% 	\item[HRC]	\angol{Human Robot Collaboration}
% 	\item[IoT]	\angol{Internet of Things}
% 	\item[IoE]	\angol{Internet of Everything}
% 	\item[TCP]	\angol{Tool Center Point}
% 	\item[HRI]	\angol{Human-Robot Interaction}
% \end{abbreviations}
%---------------------------------------------------------

%--Függelékek---------------------------------------------
\newpage
\appendix
\subfile{sections/appendix}
%---------------------------------------------------------
\end{document}