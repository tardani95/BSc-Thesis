\documentclass[../main.tex]{subfiles}
\graphicspath{{resources/}{resources/android_res}}
 
\begin{document}

\section{Androidos alkalmazás elkészítése}
    \subsection{Androidról általában}
        Az Android a világ legnépszerűbb mobil operációs rendszere. Több milliárd eszközön fut, mint például a telefonokon, órákon, táblagépeken, TV-ken, és még sok máson. 
        Különböző alakú és méretű eszközökön egyaránt elfut, ezzel óriási flexibilitást biztosítva az alkalmazás fejlesztők számára. \cite{android_about} 
        
        A nemrégiben megjelent \textit{Android Things} lehetővé teszi az okos, internetre csatlakoztatott eszközök építését, nem csak általános, hanem kereskedelmi és ipari felhasználásra is. Egy ilyen fejlesztői készletet mutat be az \ref{fig:android_things}.~ábra. A meglévő Androidos fejlesztői eszközökön kívül elérhetővé válik az alacsony szintű I/O könyvtárak kezelése is. 
        
        Azért döntöttem az Android alapú vezérlés mellett, mert megbízható, biztonságos és olcsó eszközökön is tökéletesen működik. 
        
        \begin{figure}[h!] %https://shop.technexion.com/pico-pi-imx7-startkit-rainbow-hat.html
            \centering
            \includegraphics[height=9cm]{android_res/pico-pi-imx7-startkit.jpg}
            \caption{PICO-PI-IMX7 Startkit\cite{android_starter_kit}}
            \label{fig:android_things}
        \end{figure}
    
    \subsection{Android-os szoftverfejlesztés megismerése}
        Az Android-os alkalmazások fejlesztéséhez elengedhetetlen a Java programozási nyelv, hiszen Java API keretrendszerre épít a platform. A platformmal először még 3. félévben egy villanykaros szabadon választható tárgy keretében találkoztam. Akkoriban még nem ismertem a Java-t, ezért ennek az elsajátításával kezdtem meg a tanulást\cite{b_java}. Az Android specifikus részek megismerésében nagyon nagy segítségemre volt a tantárgy maga, az Android Developer weboldal\cite{android_guides} és egy ismerősöm által ajánlott \textit{Android Programming: The Big Nerd Ranch Guide} című könyv\cite{b_android2}. Az alkalmazás fejlesztése során igyekeztem a tanultakon kívül a clean code elvek használatára\cite{android_clean_code}.
        
    % \subsection{Android Platform felépítése} %https://developer.android.com/guide/platform/
    %     Az Android egy nyílt forráskódú, Linux alapú szoftvercsomag, amely számos eszköz és formai tényező számára készült. A ~\ref{fig:android_szoftvercsomag}. ábra mutatja a platform legfontosabb összetevőit.
    %     \begin{figure}[h!]
    %         \centering
    %         \includegraphics[width=11cm]{android_res/android_szoftvercsomag.png}
    %         \caption{Android szoftvercsomag}
    %         \label{fig:android_szoftvercsomag}
    %     \end{figure}
        
    %     \subsubsection{A Linux kernel}
    %         Az Android platform alapja a Linux kernel. Többek között az Android Runtime (ART) is Linux kernelen alapszik, ami magában rejti az olyan funkciókat, mint a szálkezelés és az alacsonyszintű  memóriakezelés.
    %         A Linux kernel segítségével az Android kihasználhatja a kulcsfontosságú biztonsági funkciókat, és lehetővé teszi az eszközgyártók számára, hogy egy jól ismert rendszermaghoz hardver-illesztőprogramokat fejlesszenek ki.\cite{android_platform}
            
    %     \subsubsection{Hardware Abstraction Layer (HAL)}
    %          A hardver absztrakciós réteg (HAL) egy olyan szabványos felület, amely a hardver képességeit a magasabb szintű Java API keretrendszer számára biztosítja. A HAL több könyvtármodulból áll,  amelyek mindegyike egy adott típusú hardverösszetevőhöz, például a kamerához vagy a bluetooth modulhoz, ad hozzáférést. Amikor egy API hívás érkezik egy adott hardverhez, akkor az Android rendszer betölti a megfelelő komponenshez tartozó könyvtár-modulokat.\cite{android_platform}
            
    %     \subsubsection{Android Runtime (ART)}
    %         Az Android 5.0-s verzióját (API-szint 21) vagy újabb verziót használó eszközök esetében minden alkalmazás saját folyamatában és a saját példányán fut. Az ART olyan virtuális gépek futtatására íródott, amelyek alacsony memóriájú eszközökön futtatnak DEX-fájlokat, speciálisan az Androidra tervezett bytecode formátumot, amely a minimális memóriahasználatra lett optimalizálva. A toolchainek lefordítják a Java forrásokat DEX bytecode-ba, amelyek már futtathatók az Android platformon.\cite{android_platform}
            
    %     \subsubsection{Natív C/C++ könyvtárak}
    %         Számos fő Android rendszer komponens és szolgáltatás, mint például az ART és a HAL is natív kódokon alapuló könyvtárokon alapszanak. Az Android platform a Java keretrendszernek megadja a hozzáférést ezekhez a könyvtárakhoz. Például hozzáférhetünk az OpenGL ES-hez az Android keretrendszer Java OpenGL API-val és így 2D-s és 3D-s grafikákat készíthetünk az alkalmazásainkban.\cite{android_platform}
            
    %         Ha olyan alkalmazást fejlesztünk, amiben C/C++ kód van, akkor használhatjuk az Android NDK-t hogy a natív kódból közvetlenül hozzáférhessünk ezekhez a könyvtárakhoz.
            
    %     \subsubsection{Java API keretrendszer}
    %         A teljes Android rendszer eszköztára elérhető Java-ban írt API-kon keresztül. Ezek az API-k az építőelemek - moduláris rendszerelemek, szolgáltatások - amikből felépíthetőek az Androidos alkalmazások. A fejlesztőknek teljes hozzáférése van az Android rendszer által használt összes API-hoz.\cite{android_platform}
            
    %     \subsubsection{Rendszer alkalmazások}
    %         Az Androidba beépítve található egy sor alapvető alkalmazás az e-mailezéshez, az SMS-ezéshez, a naptárhoz, az interneten való szörfözéshez, stb. Ezeket később telepített, általunk előnyben részesített alkalmazásokkal lehet helyettesíteni. Nemcsak alkalmazásként használhatók, hanem kulcsfontosságú képességeket biztosítanak a fejlesztők számára is, amelyeket a saját alkalmazásukból elérhetnek. Például ha SMS-t szeretnénk küldeni, akkor nem kell saját magunknak leprogramozni, hanem helyette csak meghívni a már telepített SMS alkalmazások valamelyikét, hogy küldje el nekünk.\cite{android_platform}
        
    \subsection{Android alkalmazás rövid felépítése} %https://developer.android.com/guide/components/fundamentals
        Az android alkalmazások alapvetően négy fő komponensből épülnek fel, ahol a Content Provider-ek kivételével mind alkalmazásra került\cite{android_app_fund}:
        \begin{enumerate}
            \item Acitities - felületi (UI) elemek
            \item Services - háttérben futó folyamatok
            \item Content Providers - adat szolgáltók
            \item Broadcast Receivers - rendszer szintű eseményekre reagálás
        \end{enumerate}
        
        \subsubsection{Services}
            Androidos környezetben a szolgáltatások egy hosszabb ideig, háttérben futó folyamatot jelképeznek. Nem rendelkeznek felhasználói felülettel és más komponensek elindíthatják vagy vezérlés céljából rácsatlakozhatnak. Az alkalmazásomban két darab ilyen Service is megtalálható: az egyik az UDP csomagok küldésért felelős, a másik az alkalmazásba belekódolt zeneszám lejátszásáért.
        
        \subsubsection{Activities}
            Az Activity-k, pontosabban egy Activity-re csatolt Fragment-ek, alkotják a felhasználói felületem. A felhasználói felületet a manapság gyakori 5.5-inch-es, 16:9-es képarányú, FullHD (1920x1080) felbontású mobiltelefonokra terveztem és valósítottam meg, de ennél nagyobb kijelzőjű tableteken is elfut az alkalmazás. Az Activity-k feladata Service-ek vezérlése a felhasználói felület (csúszka mozgatása, gomb megnyomása), illetve a gyorsulásérzékelő és magnetométer szenzoradatok változására. 
            
        \subsubsection{Broadcast Receivers}
            Broadcast Receiver-ek segítségével iratkozhatunk fel rendszer szintű eseményekre. Ilyen események például az SMS és a bejövő telefonhívás, amiknek a bekövetkezésekor beállítható, hogy mit csináljon az alkalmazás. 
    
% \newpage
    \subsection{Az elkészült alkalmazás funkciói és felhasználói kézikönyv}
        Az alkalmazás indításakor szükséges, hogy valamilyen WiFi hálózathoz legyen csatlakoztatva az eszköz. Ekkor az
        alapértelmezett Simple Color Picker képernyő töltődik be, amit később a beállításokban módosíthatunk kedvünk szerint. Ha nincs WiFi hálózatra csatlakoztatva az eszköz, akkor egy hibát jelző oldal jelenik meg, amin a csatlakozás gombra kattintva eljuthatunk a készülék WiFi beállítás menüjébe, és ott csatlakozhatunk a hálózatra (\ref{fig:wifi_err_01}., \ref{fig:wifi_err_02}. és \ref{fig:wifi_err_03}. ábrák).
        
        %\subsubsection{Hibaelhárítás}
              
            \begin{figure}[h!]
                \begin{floatrow}
                    \ffigbox{\includegraphics[width=4.2cm]{android_res/screen_pictures/wifi_err_01.png}}{\caption{Koppintson a csatlakozás gombra!}\label{fig:wifi_err_01}}
                    \ffigbox{\includegraphics[width=4.2cm]{android_res/screen_pictures/wifi_err_02.png}}{\caption{Majd a bekapcsolásra!}\label{fig:wifi_err_02}}
                \end{floatrow}
            \end{figure}
            \begin{figure}[h!]
                \centering
                \includegraphics[width=4.2cm]{android_res/screen_pictures/wifi_err_03.png}
                \caption{Végül válasza ki azt a hálózatot amire csatlakoztatva van a LED sor!}
                \label{fig:wifi_err_03}
            \end{figure}
        
% \newpage
        \subsubsection{LED sor IP-címének a beállítása}
            A LED sor %eredeti
            IP címe két módon szerezhető meg (\ref{fig:ip_addres}. ábra):
            
            \begin{enumerate}
                \item Mobiltelefonunkon a Fing nevű alkalmazással könnyedén felderíthetjük a lokális Wifi hálózaton található eszközöket IP címükkel együtt, ami innen egyszerűen kimásolható.
                \item A helyi hálózati routerre csatlakozott eszközök listájából.
            \end{enumerate}
            
            \begin{figure}[h!]
                \includegraphics[height=7cm]{android_res/screen_pictures/ip_addr_fing}
                \includegraphics[height=7cm]{android_res/screen_pictures/ip_addr_router}
                \caption{LED sor IP címének felderítése Fing nevű alkalmazással és routerünk menüjéből}
                \label{fig:ip_addres}
            \end{figure}

        \vspace{1cm}
            Ezek után visszalépve az alkalmazásba, átnavigálva a beállítások menüre beállítható a LED sor IP címe:
            
            \begin{enumerate}
                \item Navigációs menü előhozása
                
                    Az alkalmazáson belül bármelyik képernyőről ezzel (\ref{fig:onMenu1}. és \ref{fig:onMenu2}. ábrák) a két módszerrel lehet előhozni a navigációs menüt.
                    \begin{figure}[h!]
                        \begin{floatrow}
                            \ffigbox{\includegraphics[width=3.5cm]{android_res/screen_pictures/nav_menu_01.png}}{\caption{A menü gombra való kattintással}\label{fig:onMenu1}}
                            \ffigbox{\includegraphics[width=3.5cm]{android_res/screen_pictures/nav_menu_02.png}}{\caption{˝Swipe gesture segítségével˝}\label{fig:onMenu2}}
                        \end{floatrow}
                    \end{figure}
                    
                \item Beállítások fül kiválasztása (\ref{fig:tap_settigs}. ábra)
                    \begin{figure}[h!]
                        \includegraphics[width=3.5cm]{android_res/screen_pictures/tap_settings_menu}
                        \caption{Koppintson a Beállítások fülre!}
                        \label{fig:tap_settigs}
                    \end{figure}
                    
                \item Helyi hálózati IP cím beállítása (\ref{fig:tap_local_ip}. és \ref{fig:set_local_ip}. ábrák)
                    \begin{figure}[h!]
                        \begin{floatrow}
                            \ffigbox{\includegraphics[width=3.5cm]{android_res/screen_pictures/tap_local_ip.png}}{\caption{Koppintson a helyi hálózati IP cím beállításra!}\label{fig:tap_local_ip}}
                            \ffigbox{\includegraphics[width=3.5cm]{android_res/screen_pictures/set_local_ip.png}}{\caption{Állítsa be az IP címet, majd koppintson
az Ok gombra!}\label{fig:set_local_ip}}
                        \end{floatrow}
                    \end{figure}
                
            \end{enumerate}
            
 \clearpage            
        \subsubsection{Simple Color Picker mód} %https://github.com/LarsWerkman/HoloColorPicker
            Ezen a képernyőn egy külső forrásból származó Color Picker található\cite{holo_color_picker}. A csúszkák mozgatásával tudjuk kiválasztani az adott színt, majd a \textit{Szín beállítása} gomb megnyomásával állíthatjuk be a LED sor színét (\ref{fig:s_cp_01}. és \ref{fig:s_cp_02}. ábrák).
            
            \begin{figure}[h!]
                \begin{floatrow}
                    \ffigbox{\includegraphics[width=4.2cm]{android_res/screen_pictures/s_cp_01.png}}{\caption{A csúszkák mozgatásával változtathatjuk a megjelenítendő színt}\label{fig:s_cp_01}}
                    \ffigbox{\includegraphics[width=4.2cm]{android_res/screen_pictures/s_cp_02.png}}{\caption{Szín beállítását a gombra koppintással végezhetjük el}\label{fig:s_cp_02}}
                \end{floatrow}
            \end{figure}
            
            A képernyőt személyre szabhatjuk a beállítások fülön lévő automata színválasztó funkció bekapcsolásával (\ref{fig:s_cp_03}. és \ref{fig:s_cp_04}. ábrák).
            
            \begin{figure}[hb!]
                \begin{floatrow}
                    \ffigbox{\includegraphics[width=4.2cm]{android_res/screen_pictures/s_cp_03.png}}{\caption{Automatikus színválasztás funkció bekapcsolása}\label{fig:s_cp_03}}
                    \ffigbox{\includegraphics[width=4.2cm]{android_res/screen_pictures/s_cp_04.png}}{\caption{Ezek után a csúszkákat mozgatva automatikusan változtatja a színt}\label{fig:s_cp_04}}
                \end{floatrow}
            \end{figure}
            
        \subsection{Color Palette mód}
            A gombokra való koppintással a LED sor vezérlőbe beleégetett színpaletták jeleníthetők meg (\ref{fig:color_palettes}. ábra).
            \begin{figure}[h!]
                \centering
                    \includegraphics[width=4.3cm]{android_res/screen_pictures/color_palette_01}
                    \includegraphics[width=4.3cm]{android_res/color_palette_11}
                \caption{A \textit{Party Palette} gomb megnyomására megjelenő minta a LED soron}
                \label{fig:color_palettes}
            \end{figure}
        
% \newpage        
        \subsection{Party mód}
            Ebben a módban a telefon orientációjával állítható be a LED sor színe (\ref{fig:party_mode}. ábra), vagyis ha táncolunk vagy ugrálunk akkor aszerint változik a szín. A fejlesztői beállításokban ki lehet választani, hogy 
            melyik tengelyek körüli orientáció, melyik komponensnek feleljen meg a HSV (Hue Saturation Value) színtérből.
            \begin{figure}[h!]
                \centering
                    \includegraphics[height=6.8cm]{android_res/screen_pictures/phone_orientation_01}
                    \includegraphics[height=6.8cm]{android_res/screen_pictures/phone_orientation_11}
                    \includegraphics[height=6.8cm]{android_res/screen_pictures/phone_orientation_02}
                    \includegraphics[height=6.8cm]{android_res/screen_pictures/phone_orientation_22}
                \caption{A telefon orientációja és a hatására megjelenő színek}
                \label{fig:party_mode}
            \end{figure}
    \vspace{0.8cm}
%  \newpage           
        \subsection{Audio Visualizer}
            Az alkalmazásba hardcode-olt zeneszámot a Start/Stop gomb megnyomásával indítható el. Az zene elindításával körkörösen mozgó mély, közép, és magas értékek találhatóak, rendre a kör középpontjától kifelé haladva (\ref{fig:audio_visualizer}. ábra). Ezen adatokat az Androidba beépített Visualizer API FFT (Fast Fourier Transform) mintakódjával nyerem ki. A mély értékekből számítom a HSV színtér Value értékeit, ezzel a LED sor elsötétedését és felvillanását állítva. A megjelenítendő szín (Hue) a magas és mély adatok összekombinálásával határozom meg. 
            \begin{figure}[h!]
                \centering
                    \includegraphics[width=4.2cm]{android_res/screen_pictures/audio_vis_01}
                    \includegraphics[width=4.2cm]{android_res/screen_pictures/audio_vis_02}
                \caption{Szín változtatás a zene spektruma alapján}
                \label{fig:audio_visualizer}
            \end{figure}
%  \newpage       
        \subsection{Beállítások menü}
            Ezen a nézeten lehet testre szabni az alkalmazást.
%  \newpage           
            \subsubsection{Helyi hálózati IP cím beállítása (\ref{fig:tap_local_ip2}. és \ref{fig:set_local_ip2}. ábra)}
                \begin{figure}[h!]
                    \begin{floatrow}
                        \ffigbox{\includegraphics[width=4.2cm]{android_res/screen_pictures/tap_local_ip.png}}{\caption{Koppintson a helyi hálózati IP cím beállításra!}\label{fig:tap_local_ip2}}
                        \ffigbox{\includegraphics[width=4.2cm]{android_res/screen_pictures/set_local_ip.png}}{\caption{Állítsa be az IP címet, majd koppintson az Ok gombra!}\label{fig:set_local_ip2}}
                    \end{floatrow}
                \end{figure}
% \newpage            
            \subsubsection{Vizuális értesítők bekapcsolása}
                Ennek funkciónak a használatához el kell fogadni a megfelelő engedélyeket, különben nem fog működni.\\[3mm]
                Amennyiben az engedélyeket nem adjuk meg, egy hiba üzenet ugrik fel (\ref{fig:permission_denied}. ábra).\\[3mm]
                Ha mégis használni szeretnénk ezt a funkciót, akkor újra be kell állítani, majd ezt követően megadni meg az engedélyeket.\\[3mm]
                A vizuális értesítések bekapcsolása után további két beállítás válik elérhetővé, amikkel testre lehet szabni, hogy hívás, vagy sms érkezése esetén milyen színnel villogjon a LED sor (~\ref{fig:vis_not_01}. ábra).\\[3mm]
                A bejövő hívás vagy SMS szín opcióra koppintva előhozhatunk egy ablakot (\ref{fig:vis_not_03}. ábra), ahol beállíthatjuk a kívánt értesítési színt.
                
                \begin{figure}[ht!]
                    \centering
                    \includegraphics[width=3.6cm]{android_res/screen_pictures/visual_notification_02.png}
                    \caption{Felugró hibaüzenet}
                    \label{fig:permission_denied}
                \end{figure}
                % Ha mégis használni szeretnénk ezt a funkciót, akkor újra be kell állítani, majd ezt követően megadni meg az engedélyeket.\\[3mm]
                % A vizuális értesítések bekapcsolása után további két beállítás válik elérhetővé, amikkel testre lehet szabni, hogy hívás, vagy sms érkezése esetén milyen színnel villogjon a LED sor (~\ref{fig:vis_not_01}. ábra).\\[3mm]
                % A bejövő hívás vagy SMS szín opcióra koppintva előhozhatunk egy ablakot (\ref{fig:vis_not_03}. ábra), ahol beállíthatjuk a kívánt értesítési színt.
                
                \begin{figure}[ht!]
                    \includegraphics[height=6.1cm]{android_res/screen_pictures/visual_notification_01.png}
                    \caption{Az engedélyek elfogadása után elérhető két új beállítási lehetőség}
                     \label{fig:vis_not_01}
                \end{figure} 
                
                % A bejövő hívás vagy SMS szín opcióra koppintva előhozhatunk egy ablakot (\ref{fig:vis_not_03}. ábra), ahol beállíthatjuk a kívánt értesítési színt. 
                
                \begin{figure}[h!]
                    \includegraphics[width=3.6cm]{android_res/screen_pictures/visual_notification_03.png}
                    \caption{Bal oldalt a jelenlegi, jobb oldalt a beállítani kívánt szín látható, amit az Ok gomb lenyomásával ment el az alkalmazás.}
                    \label{fig:vis_not_03}
                \end{figure}
            
            \newpage           
            \subsubsection{Automata színbeállítás a Simple Color Picker módhoz}
                Az automata színválasztás bekapcsolása és a hatására változó Simple Color Picker mód ablak (~\ref{fig:s_cp_05}. és ~\ref{fig:s_cp_06}. ábrák).
                \begin{figure}[h!]
                \begin{floatrow}
                    \ffigbox{\includegraphics[width=4.4cm]{android_res/screen_pictures/s_cp_03.png}}{\caption{Automatikus színválasztás funkció bekapcsolása}\label{fig:s_cp_05}}
                    \ffigbox{\includegraphics[width=4.4cm]{android_res/screen_pictures/s_cp_04.png}}{\caption{Ezek után a csúszkákat mozgatva automatikusan változtatja a színt}\label{fig:s_cp_06}}
                \end{floatrow}
            \end{figure}
                
\end{document}
