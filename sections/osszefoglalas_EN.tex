\documentclass[../main.tex]{subfiles}
 
\begin{document}
\phantomsection
\addcontentsline{toc}{section}{Summary}
\section*{Summary}
\markright{\MakeUppercase{Summary}}
    The goal of my thesis was to develop an android controlled wireless LED-lighting system. First, I researched the similar products already available on the market and set up my requirements. I accomplished the hardware development in three main phases (from the breadboard circuit until the manufactured PCB). 
    I created a 3D modell of the enclosure housing of the board with the Autodesk Fusion 360 program and 3D printed it on my own Prusa i3 MK3 3D printer. I soldered and tested the PCBs and corrected the inaccuracies. 
    In the meantime, I implemented the embedded software and the Android application. I configured the ESP8266 Wifi module with AT commands and connected it with the LED strip controller board. Finally, I installed the finished product on our teracce and in my dorm room.
    
    During my thesis, I learned how to use the Git version control system and GitHub, the hosting solution for Git repositories. 
    I expanded my knowledge in the object oriented Java and Android programming. While writing the program, I attempted to use the language specific naming conventions, the developer guidelines and the clean code principles. I became familiar with the working mechanisms of microcontrollers, also on the level of registers. From the datasheets, reference manuals and application notes, I was able to implement my own embedded software.
    I learned how to plan and develop electrical circuits. According to the datasheets, I created the schematic layout. Corresponding to the layout design guidelines and the top-10-tips-by-electrical-engineers, I placed the components and routed the PCB, which was manufactured by JLCPCB. I refreshed my 3D modelling knowledge and completed it with parametric modelling skills. Lastly, I got more experienced with soldering and using an oscilloscope during the measurements and error correction.
\end{document}