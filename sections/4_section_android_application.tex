\documentclass[../main.tex]{subfiles}
\graphicspath{{resources/}{resources/android_res}}
 
\begin{document}

\section{Androidos alkalmazás elkészítése}
    \subsection{Androidról általában}
        Android a világ legnépszerűbb mobil operációs rendszere. Több milliárd eszközön fut, mint például a telefonokon, órákon, táblagépeken, TV-ken, és még sok máson. 
        Különböző alakú és méretű eszközökön egyaránt elfut, ezzel óriási flexibilitást biztosítva az alkalmazás fejlesztők számára. 
        A nemrégiben megjelent \textit{Android Things} lehetővé teszi az okos, internetre csatlakoztatott eszközök építését, nem csak általános, hanem kereskedelmi és ipari felhasználásra is. A meglévő Androidos fejlesztői eszközökön kívül elérhetővé válik az alacsony szintű I/O könyvtárak kezelése is. 
        
        Azért döntöttem az Android alapú vezérlés mellett, mert megbízható, biztonságos, mindezek mellett nagyon olcsó eszközökön is tökéletesen működik. 
        
        \begin{figure}[h!]
            \centering
            \includegraphics[width=7cm]{android_res/android_logo.png}
            \caption{Android logó}
            \label{fig:android_logo}
        \end{figure}
        
    \subsection{Android Platform felépítése} %https://developer.android.com/guide/platform/
        Az Android egy nyílt forráskódú, Linux alapú szoftvercsomag, amely számos eszköz és formai tényező számára készült. A ~\ref{fig:android_szoftvercsomag}. ábra mutatja a platform legfontosabb összetevőit.
        \begin{figure}[h!]
            \centering
            \includegraphics[width=11cm]{android_res/android_szoftvercsomag.png}
            \caption{Android szoftvercsomag}
            \label{fig:android_szoftvercsomag}
        \end{figure}
        
        \subsubsection{A Linux kernel}
            Az Android platform alapja a Linux kernel. Például az Android Runtime (ART) a Linux kernelen alapszik, ami magában rejti az olyan funkciókat, mint a szálkezelés és az alacsonyszintű  memóriakezelés.
            A Linux kernel segítségével az Android kihasználhatja a kulcsfontosságú biztonsági funkciókat, és lehetővé teszi az eszközgyártók számára, hogy egy jól ismert rendszermaghoz hardver-illesztőprogramokat fejlesszenek ki.
            
        \subsubsection{Hardware Abstacion Layer (HAL)}
             A hardver absztrakciós réteg (HAL) olyan szabványos felületet, amely a hardver képességeit a magasabb szintű Java API keretrendszer számára biztosítja. A HAL több könyvtármodulból áll,  amelyek mindegyike egy adott típusú hardverösszetevőhöz, például a kamerához vagy a bluetooth modulhoz ad hozzáférést. Amikor egy API hívás érkezik egy adott hardverhez, akkor az Android rendszer betölti a megfelelő komponenshez tartozó könyvtár-modulokat.
            
        \subsubsection{Android Runtime (ART)}
            Az Android 5.0-s verzióját (API-szint 21) vagy újabb verziót használó eszközök esetében minden alkalmazás saját folyamatában és az saját példányán fut. Az ART olyan virtuális gépek futtatására íródott, amelyek alacsony memóriájú eszközökön futtatnak DEX-fájlokat, speciálisan az Androidra tervezett bytecode formátumot, amely a minimális memóriahasználatra optimalizált. A toolchainek lefordítják a Java forrásokat DEX bytecode-ba, amelyek már futtathatóak az Android platformon.
            
        \subsubsection{Natív C/C++ könyvtárak}
            Számos fő Android rendszer komponens és szolgáltatás, mintpéldául az ART és a HAL, amelyek natív kódokon alapuló könyvtárokon alapszanak. Az Android platform a Java keretrendszernek megadja a hozzáférést ezekhez a könyvtárakhoz. Például hozzáférhetünk az OpenGL ES-hez az Android keretrendszer Java OpenGL API-val és így 2D-s és 3D-s grafikákat készíthetünk az alkalmazásainkban.
            
            Ha olyan alkalmazást fejlesztünk, amiben C/C++ kód van, akkor használhatjuk az Android NDK-t hogy a natív kódból közvetlenül hozzáférhessünk ezekhez a könyvtárakhoz.
            
        \subsubsection{Java API keretrendszer}
            
            
        \subsubsection{Rendszer alkalmazások}
            
        
    \subsection{Android alkalmazások felépítése} %https://developer.android.com/guide/components/fundamentals
        There are four different types of app components:

            Activities
            
            Services
            
            Broadcast receivers
            
            Content providers

        
    
        
        Felhasználói felület
        A felhasználói felület a manapság gyakori 5.5-inch-es, 16:9-es képarányú, FullHD (1920x1080) felbontású mobiltelefonokra terveztem és valósítottam meg, de ennél nagyobb kijelzőjű tableteken is elfut az alkalmazás.
        
        
%https://shop.technexion.com/pico-pi-imx7-startkit-rainbow-hat.html
%https://developer.android.com/about/
\end{document}