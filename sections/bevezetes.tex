\documentclass[../main.tex]{subfiles}
 
\begin{document}
\section{Bevezetés}
    \subsection{Célkitűzések}
        A célom egy olyan feladat megvalósítása volt, amely során képes leszek komplex rendszerek tervezésére, értelmezésére, illetve széleskörű tudásra tehetek szert. Több közül végül még egy kollégiumi szobában megfogalmazott feladat mellett döntöttem. 
        Az elképzelésünk az volt, hogy az Ebay-en fellelhető olcsó LED-szalagokhoz készítsek egy olyan hardvert, amit az én és a szobatársam Android-os mobiltelefonjával tudunk vezérelni, illetve állapotokat megjeleníteni (például hívás, SMS érkezését) Wifi-n keresztül. Továbbá szerettük volna, hogy zenére és a mobiltelefon mozgatására is tudjon villogni. 
        
        % Ezen szakdolgozat célja, hogy a képzésem során tanultakat elmélyítsem, gyakorlatba ültessem, és a gólyakori álmot valóra váltsam. 
        % A munkám tartalmazza az elektronikai tervezést, alkatrészek méretezését, a NYÁK legyártatását, az elektronikai komponensek beültetését, beforrasztását, ezek védelméül szolgáló doboz megtervezését, 3D nyomtatását, a beágyazott szoftver megírását, külső hardverrel való összekötését, illetve az egész rendszer vezérléséért felelős Android-os alkalmazást is.
        
        A szakdolgozat írásakor, kivitelezésekor törekedtem az átlátható logikus gondolkodásra, szerkezeti tagolásra. Programozásnál igyekeztem a clean code elvek betartására, és az adott nyelvnek megfelelő elnevezési konvenciók alkalmazására \cite{b_embedded_c_coding}\cite{b_programming_embedded_systems}\cite{android_clean_code}.
        
    \subsection{Áttekintés}
        %Az első fejezetben ismertetem az általam készített eszköz háztartásbeli, ipari felhasználhatóságát. Bemutatom az interneten fellelhető, forgalomban lévő eszközöket, és rávilágítok a hasonlóságokra, különbségekre.
        Az első fejezetben ismertetem az interneten fellelhető, forgalomban lévő LED-es eszközöket, rendszereket.
        A második fejezetben kitérek az elektronikai tervezésre, megvalósításra és a védődoboz elkészítésére.
        A hardver után beszámolok a beágyazott szoftver implementációjáról és a külső eszközökkel való kommunikációról.
        A következő részben magáról az Android-ról, illetve az alkalmazás szerkezetéről fogok írni. Végül egy felhasználási útmutatón keresztül ismertetem az alkalmazás működését.
\end{document}

